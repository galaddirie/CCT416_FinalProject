% Document class and import any packages that are used.
\documentclass[12pt,titlepage]{article}
\pagestyle{plain}
\usepackage[utf8]{inputenc}
\usepackage[empty]{fullpage}
\usepackage[a4paper,margin=1in, headsep=24pt, headheight=2cm]{geometry}
\usepackage{enumitem}
\usepackage{hyperref}
\usepackage{setspace}
\usepackage{indentfirst}
\usepackage{wrapfig}
\usepackage{natbib}
\usepackage{multirow}
\usepackage{titling}
\usepackage{fancyhdr}
\usepackage{graphicx}
\usepackage{titlesec}

% Set section format
\titleformat{\section}
  {\normalfont\normalsize\bfseries\itshape}{\thesection.}{1em}{}

\pagestyle{fancy}
\fancyhf{}
\renewcommand{\headrulewidth}{0pt}
\setlength{\footskip}{15pt}
\fancyfoot[R]{\thepage}

\onehalfspacing
\newcommand{\resumeSubheading}[4]{
    \begin{tabular*}{0.96\textwidth}{l@{\extracolsep{\fill}}r}
      \textbf{#1} & #2 \\
      \textit{#3} & \textit{#4} \\
    \end{tabular*}\vspace{1pt}
}

\newcommand{\resumeItemListStart}{\begin{itemize}}
\newcommand{\resumeItemListEnd}{\end{itemize}\vspace{-20pt}}

\newcommand{\resumeItem}[1]{
  \item{
    {#1 \vspace{-10pt}}
  }
}

\begin{document}

% Set page style and layout style.
\begin{titlepage}
\title{Rhetoric of Climate Change Discourse on Twitter}
\author{Nianhui Guo (1005205779), Jolene Milne (1006310902),\\
Robin Thapa (1007040355), Ryan Wong (1004438156)\\
CCT416 Social Data Analytics\\
Dr. Radha Maharaj\\
Institute of Communication, Culture, Information and Technology\\
University of Toronto Mississauga\\}
\date{4 April 2023}
\maketitle
\end{titlepage}

\newpage
\tableofcontents
\setcounter{secnumdepth}{0}

\newpage
\begin{center}
    \textbf{\Large Rhetoric of Climate Change Discourse on Twitter}
\end{center}

\section{Introduction}
The topic we chose to go with is climate change. The question that motivated our choice in this topic was curiosity behind the types of discussions, thoughts and opinions about climate change that users on social media may post online. More specifically, we also wanted to discover what the average user is posting about climate change, and the general opinions/information that is circulating around social media based on the topic of climate change. The general consensus is that the majority of people both online and offline perceive and hold opinions of climate change that are massively different from each other, whether the reason being they are uninformed, misinformed or any other possible reason. For this goal, we have used and compared various methods of data collection and sentiment analysis to answer our question as accurately as possible. The results of this analysis display the variety of user opinion and the most common sentiments/topics resulting from the data collection, bringing us closer to this goal.

\newpage
\section{Methodology}


\newpage
\section{Observations}


\newpage
\section{Discussion}


\newpage
\section{Conclusion}


\newpage
\bibliographystyle{apalike}
\bibliography{references}

\end{document}